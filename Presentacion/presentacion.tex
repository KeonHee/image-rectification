\documentclass[aspectratio=169,14pt,spanish]{beamer}

\usepackage[utf8]{inputenc}
\usepackage[T1]{fontenc}
\usepackage[spanish]{babel} % Changes theorem->teorema, proof->demostración...

%%%%%%%%%%%%%%%%%%%%%%%%%%%%%%%%%%%%%%%%%%%%%%%
%%%%%%%%%%%%%%%% CONFIGURATION %%%%%%%%%%%%%%%%
%%%%%%%%%%%%%%%%%%%%%%%%%%%%%%%%%%%%%%%%%%%%%%%

\usetheme{default}
\setbeamertemplate{navigation symbols}{} % Removes the navigation symbols :_)

% Opens the PDF document in full screen by default :)
%\hypersetup{pdfpagemode=FullScreen}

%Language
\selectlanguage{spanish}

%%%%%%%%%%%%%%%%%%%%%%%%%%%%%%%%%%%%%%%%%%%%%%%
%%%%%%%%%%%%%%%%%%%% TITLE %%%%%%%%%%%%%%%%%%%%
%%%%%%%%%%%%%%%%%%%%%%%%%%%%%%%%%%%%%%%%%%%%%%%

\title[Rectificación]{Rectificación de un par estéreo}
\subtitle[Implementación de Loop Zhang]{Implementación del algoritmo de Loop \& Zhang}
\author[A. A. Caballero, A. G. Montoro]{Antonio Álvarez Caballero, Alejandro García Montoro}
\institute[UGR]{Universidad de Granada}
\date{\today}

% Frame whenever a new section begins
\AtBeginSection[]{
    \begin{frame}[plain]
        \begin{center}
            \huge{\insertsection}
        \end{center}
    \end{frame}
}

%%%%%%%%%%%%%%%%%%%%%%%%%%%%%%%%%%%%%%%%%%%%%%%
%%%%%%%%%%%%%%%%%%% COLORS %%%%%%%%%%%%%%%%%%%%
%%%%%%%%%%%%%%%%%%%%%%%%%%%%%%%%%%%%%%%%%%%%%%%

%%%%%%%%%%%%%%%% DECLARATIONS %%%%%%%%%%%%%%%%%

\definecolor{c_gray_l}{HTML}{FAFAFA} % background light gray
\definecolor{c_metal}{HTML}{37474F}  % background dark metal blue

%%%%%%%%%%%%%%%% DEFINITIONS %%%%%%%%%%%%%%%%%*

\setbeamercolor*{default}{
    bg=c_metal,
    fg=c_gray_l
}

\setbeamercolor*{palette primary}{
    parent=default
}

\setbeamercolor*{palette secondary}{
    bg=c_gray_l,
    fg=c_metal
}

\setbeamercolor*{background canvas}{
    parent = palette primary
}

\setbeamercolor*{normal text}{
    fg = c_gray_l
}

\setbeamercolor*{titlelike}{
    parent = palette secondary
}

\setbeamercolor*{section in toc}{
    parent = palette primary
}

\setbeamercolor*{block title alerted}{
    bg = red!60,
    fg = c_gray_l
}

\setbeamercolor*{block body alerted}{
    parent = palette secondary
}

%%%%%%%%%%%%%%%%%%%%%%%%%%%%%%%%%%%%%%%%%%%%%%%
%%%%%%%%%%%%%%%%%% DOCUMENT %%%%%%%%%%%%%%%%%%%
%%%%%%%%%%%%%%%%%%%%%%%%%%%%%%%%%%%%%%%%%%%%%%%

\begin{document}

    \titlepage

    \begin{frame}[t]{Contenidos}
        \tableofcontents
    \end{frame}
    %--- Next Frame ---%

    \section{Descripción del problema}

    \begin{frame}{Descripción del problema}{}
        %\alert{Una alerta}

        \begin{itemize}
            \item Rectificación: Aplicar homografías a par estéreo para que las líneas
              epipolares queden horizontales y paralelas.
            \item Búsqueda de correspondencias eficiente: Sólo buscamos en una fila
              horizontal de píxeles.
            \item Mapas de disparidad, reconstrucción de profundidad.
            \item El método propuesto por \emph{Loop \& Zhang} no precisa de un sistema
              de cámaras calibrado.
            \item Minimizar la distorsión.
        \end{itemize}

        % \begin{block}{Título de bloque}
        %     Texto en bloque
        % \end{block}
    \end{frame}
    %--- Next Frame ---%

    \section{Detalles de la implementación}

    \begin{frame}{Detalles de la implementación}{}
      Descomposición de las homografías en transformaciones sencillas.
      Suponiendo que conocemos la geometría epipolar, calculamos:
        \begin{enumerate}
            \item Transformación proyectiva.
            \item Tranformación de semejanza.
            \item Transformación de cizalla
        \end{enumerate}

        La homografía final queda $H_s * H_r * H_p$

        % \begin{alertblock}{Título de bloque alerta}
        %     Texto en bloque alerta
        % \end{alertblock}
    \end{frame}
    %--- Next Frame ---%

    \subsection{Transformación proyectiva}

      %Cálculo de transformación proyectiva

      \subsubsection{Primera aproximación a la raíz}

        \begin{frame}{Transformación proyectiva}{Primera aproximación a la raíz}
            % \begin{exampleblock}{Título de bloque ejemplo}
            %     Texto en bloque ejemplo
            % \end{exampleblock}

            \begin{description}[<+->]
                \item [Primer ítem] Descripción 1
                \item [Segundo ítem] Descripción 2
                \item [Tercer ítem] Descripción 3
            \end{description}
        \end{frame}
        %--- Next Frame ---%
      \subsubsection{Optimización de la raíz con Newton-Raphson}

        \begin{frame}{Transformación proyectiva}{Optimización de la raíz con Newton-Raphson}
            \begin{overlayarea}{\textwidth}{3cm}
                \only<1> {Lorem ipsum dolor sit amet.}
                \only<2->{Fusce pretium ullamcorper neque sit amet luctus.}
            \end{overlayarea}
        \end{frame}
        %--- Next Frame ---%
        \subsubsection{Construcción de la proyección}

          \begin{frame}{Transformación proyectiva}{Construcción de la proyección}
              % \begin{exampleblock}{Título de bloque ejemplo}
              %     Texto en bloque ejemplo
              % \end{exampleblock}

              \begin{description}[<+->]
                  \item [Primer ítem] Descripción 1
                  \item [Segundo ítem] Descripción 2
                  \item [Tercer ítem] Descripción 3
              \end{description}
          \end{frame}
      \subsection{Transformación de semejanza}

        \begin{frame}{Transformación de semejanza}
          Cálculo de semejanza

        \end{frame}

      \subsection{Transformación de cizalla}

        \subsubsection{Reducción de distorsión}
          \begin{frame}{Transformación de cizalla}{Reducción de distorsión}
              % \begin{exampleblock}{Título de bloque ejemplo}
              %     Texto en bloque ejemplo
              % \end{exampleblock}

              \begin{description}[<+->]
                  \item [Primer ítem] Descripción 1
                  \item [Segundo ítem] Descripción 2
                  \item [Tercer ítem] Descripción 3
              \end{description}
          \end{frame}

        \subsubsection{Escalado y traslación}
          \begin{frame}{Transformación de cizalla}{Escalado y traslación}
              % \begin{exampleblock}{Título de bloque ejemplo}
              %     Texto en bloque ejemplo
              % \end{exampleblock}

              \begin{description}[<+->]
                  \item [Primer ítem] Descripción 1
                  \item [Segundo ítem] Descripción 2
                  \item [Tercer ítem] Descripción 3
              \end{description}
          \end{frame}




    \section{Pruebas y valoración de resultados}

      \begin{frame}{Pruebas y valoración de resultados}{Prueba 1}


      \end{frame}

      \begin{frame}{Pruebas y valoración de resultados}{Prueba 2}


      \end{frame}

      \begin{frame}{Pruebas y valoración de resultados}{Prueba 3}


      \end{frame}

    \section{Conclusiones}

    \begin{frame}{Conclusiones}{Algo?}


    \end{frame}

    \begin{frame}

      Preguntas?
    \end{frame}

\end{document}
