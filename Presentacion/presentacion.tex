\documentclass[aspectratio=169,14pt,spanish]{beamer}

\usepackage[utf8]{inputenc}
\usepackage[T1]{fontenc}
\usepackage[spanish]{babel} % Changes theorem->teorema, proof->demostración...
\usepackage{tikz}

%%%%%%%%%%%%%%%%%%%%%%%%%%%%%%%%%%%%%%%%%%%%%%%
%%%%%%%%%%%%%%%% CONFIGURATION %%%%%%%%%%%%%%%%
%%%%%%%%%%%%%%%%%%%%%%%%%%%%%%%%%%%%%%%%%%%%%%%

\usetheme{default}
\setbeamertemplate{navigation symbols}{} % Removes the navigation symbols :_)

% Opens the PDF document in full screen by default :)
\hypersetup{pdfpagemode=FullScreen}

%Language
\selectlanguage{spanish}

%%%%%%%%%%%%%%%%%%%%%%%%%%%%%%%%%%%%%%%%%%%%%%%
%%%%%%%%%%%%%%%%%%%% TITLE %%%%%%%%%%%%%%%%%%%%
%%%%%%%%%%%%%%%%%%%%%%%%%%%%%%%%%%%%%%%%%%%%%%%

\title[Rectificación]{Rectificación de un par estéreo}
\subtitle[Implementación de Loop Zhang]{Implementación del algoritmo de Loop Zhang}
\author[A. A. Caballero, A. G. Montoro]{Antonio Álvarez Caballero, Alejandro García Montoro}
\institute[UGR]{Universidad de Granada}
\date{\today}

%%%%%%%%%%%%%%%%%%%%%%%%%%%%%%%%%%%%%%%%%%%%%%%
%%%%%%%%%%%%%%%%%%% COLORS %%%%%%%%%%%%%%%%%%%%
%%%%%%%%%%%%%%%%%%%%%%%%%%%%%%%%%%%%%%%%%%%%%%%

%%%%%%%%%%%%%%%% DECLARATIONS %%%%%%%%%%%%%%%%%

\definecolor{c_gray_l}{HTML}{FAFAFA} % background light gray
\definecolor{c_metal}{HTML}{37474F}  % background dark metal blue

%%%%%%%%%%%%%%%% DEFINITIONS %%%%%%%%%%%%%%%%%*

\setbeamercolor*{default}{
    bg=c_metal,
    fg=c_gray_l
}

\setbeamercolor*{palette primary}{
    parent=default
}

\setbeamercolor*{palette secondary}{
    bg=c_gray_l,
    fg=c_metal
}

\setbeamercolor*{background canvas}{
    parent = palette primary
}

\setbeamercolor*{normal text}{
    fg = c_gray_l
}

\setbeamercolor*{titlelike}{
    parent = palette secondary
}

\setbeamercolor*{section in toc}{
    parent = palette primary
}

\setbeamercolor*{block title alerted}{
    bg = red!60,
    fg = c_gray_l
}

\setbeamercolor*{block body alerted}{
    parent = palette secondary
}

\newcommand{\btVFill}{\vskip0pt plus 1filll}

%%%%%%%%%%%%%%%%%%%%%%%%%%%%%%%%%%%%%%%%%%%%%%%
%%%%%%%%%%%%%%%%% TEMPLATES %%%%%%%%%%%%%%%%%%%
%%%%%%%%%%%%%%%%%%%%%%%%%%%%%%%%%%%%%%%%%%%%%%%

% Frame whenever a new section begins
\AtBeginSection[]{
    \begin{frame}[plain]
        \begin{center}
            \huge{\insertsection}
        \end{center}
    \end{frame}
}

\setbeamertemplate{title page}{
\begin{frame}
    \begin{tikzpicture}[remember picture,overlay]
        \node (title) [shape=rectangle, fill=white, minimum height=20mm, minimum width=\paperwidth, anchor=north west] at ([yshift=-8mm] current page.north west) {};
        \node at (title.center) {\color{c_metal} \huge \inserttitle};

        \node (subtitle) [shape=rectangle, minimum width=\paperwidth, anchor=west] at ([yshift=5.9cm]current page.south west) {
			\textsc{\insertsubtitle}
		};
    \end{tikzpicture}
    
    \begin{center}
        \insertauthor
    \end{center}
\end{frame}
}

%%%%%%%%%%%%%%%%%%%%%%%%%%%%%%%%%%%%%%%%%%%%%%%
%%%%%%%%%%%%%%%%%% DOCUMENT %%%%%%%%%%%%%%%%%%%
%%%%%%%%%%%%%%%%%%%%%%%%%%%%%%%%%%%%%%%%%%%%%%%

\begin{document}

    \titlepage

    \begin{frame}[t]{Contenidos}
        \tableofcontents
    \end{frame}
    %--- Next Frame ---%

    \section{Primera sección}

    \begin{frame}{Título 1}{Subtítulo 1.1}
        \alert{Una alerta}

        \begin{itemize}
            \item Primer ítem
            \item Segundo ítem
            \item Tercer ítem
        \end{itemize}

        \begin{block}{Título de bloque}
            Texto en bloque
        \end{block}
    \end{frame}
    %--- Next Frame ---%

    \section{Segunda sección}

    \begin{frame}{Título 2}{Subtítulo 2.1}
        \begin{enumerate}
            \item Primer ítem
            \item Segundo ítem
            \item Tercer ítem
        \end{enumerate}

        \begin{alertblock}{Título de bloque alerta}
            Texto en bloque alerta
        \end{alertblock}
    \end{frame}
    %--- Next Frame ---%

    \subsection{Subsección}

    \begin{frame}{Título 2}{Subtítulo 2.2}
        \begin{exampleblock}{Título de bloque ejemplo}
            Texto en bloque ejemplo
        \end{exampleblock}

        \begin{description}[<+->]
            \item [Primer ítem] Descripción 1
            \item [Segundo ítem] Descripción 2
            \item [Tercer ítem] Descripción 3
        \end{description}
    \end{frame}
    %--- Next Frame ---%

    \begin{frame}{Título 2}{Subtítulo 2.3}
        \begin{overlayarea}{\textwidth}{3cm}
            \only<1> {Lorem ipsum dolor sit amet.}
            \only<2->{Fusce pretium ullamcorper neque sit amet luctus.}
        \end{overlayarea}
    \end{frame}
    %--- Next Frame ---%

\end{document}
