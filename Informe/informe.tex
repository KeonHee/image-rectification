\documentclass[a4paper, 11pt]{article}

%Comandos para configurar el idioma
\usepackage[spanish,activeacute]{babel}
\usepackage[utf8]{inputenc}
\usepackage[T1]{fontenc} %Necesario para el uso de las comillas latinas.
\usepackage{geometry}
\usepackage{graphicx}

%Importante que esta sea la última órden del preámbulo
\usepackage{hyperref}

\newcommand\fnurl[2]{%
  \href{#2}{#1}\footnote{\url{#2}}%
}

\newcommand{\includecode}[2][c]{\lstinputlisting[caption=#2, escapechar=, style=custom#1]{#2}<!---->}

%Paquetes matemáticos
\usepackage{amsmath,amsfonts,amsthm}
\usepackage[all]{xy} %Para diagramas
\usepackage{enumerate} %Personalización de enumeraciones
\usepackage{tikz} %Dibujos

%Tipografía escalable
\usepackage{lmodern}
%Legibilidad
\usepackage{microtype}

%Código
\usepackage{listings}
\usepackage{color}

\definecolor{dkgreen}{rgb}{0,0.6,0}
\definecolor{gray}{rgb}{0.5,0.5,0.5}
\definecolor{mauve}{rgb}{0.58,0,0.82}

\definecolor{listinggray}{gray}{0.9}
\definecolor{lbcolor}{rgb}{0.9,0.9,0.9}
\lstset{
backgroundcolor=\color{lbcolor},
    tabsize=4,
rulecolor=,
    language=[GNU]C++,
        basicstyle=\scriptsize,
        upquote=true,
        aboveskip={1.5\baselineskip},
        columns=fixed,
        showstringspaces=false,
        extendedchars=false,
        breaklines=true,
        prebreak = \raisebox{0ex}[0ex][0ex]{\ensuremath{\hookleftarrow}},
        frame=single,
        numbers=left,
        showtabs=false,
        showspaces=false,
        showstringspaces=false,
        identifierstyle=\ttfamily,
        keywordstyle=\color[rgb]{0,0,1},
        commentstyle=\color[rgb]{0.026,0.112,0.095},
        stringstyle=\color[rgb]{0.627,0.126,0.941},
        numberstyle=\color[rgb]{0.205, 0.142, 0.73},
%        \lstdefinestyle{C++}{language=C++,style=numbers}’.
}
\lstset{
    backgroundcolor=\color{lbcolor},
    tabsize=4,
  language=C++,
  captionpos=b,
  tabsize=3,
  frame=lines,
  numbers=left,
  numberstyle=\tiny,
  numbersep=5pt,
  breaklines=true,
  showstringspaces=false,
  basicstyle=\footnotesize,
%  identifierstyle=\color{magenta},
  keywordstyle=\color[rgb]{0,0,1},
  commentstyle=\color[rgb]{0.026,0.112,0.095}, % Darkgreen
  stringstyle=\color{red}
  }
% Slightly bigger margins than the latex defaults

\geometry{verbose,tmargin=1in,bmargin=1in,lmargin=1in,rmargin=1in}
\setlength{\parskip}{.5em} % por defecto el espacio entre párrafos es 0pt

\theoremstyle{definition}
\newtheorem{ejercicio}{Ejercicio}
\newtheorem{cuestion}{Cuestión}
\newtheorem*{solucion}{Solución}
\newtheorem*{bonus}{BONUS}

%%%%%%%% New sqrt
\usepackage{letltxmacro}
\makeatletter
\let\oldr@@t\r@@t
\def\r@@t#1#2{%
\setbox0=\hbox{$\oldr@@t#1{#2\,}$}\dimen0=\ht0
\advance\dimen0-0.2\ht0
\setbox2=\hbox{\vrule height\ht0 depth -\dimen0}%
{\box0\lower0.4pt\box2}}
\LetLtxMacro{\oldsqrt}{\sqrt}
\renewcommand*{\sqrt}[2][\ ]{\oldsqrt[#1]{#2} }
\makeatother

%%%%%%%%%%%%%%%%%%%%%%%%%%%%%%%%%%%%%%%%%%%

\hypersetup{
  pdftitle={Informe de proyecto: Implementación del algoritmo de rectificación de Loop \& Zhang},
  pdfauthor={Antonio Álvarez Caballero, Alejandro García Montoro},
  unicode,
  breaklinks=true,  % so long urls are correctly broken across lines
  colorlinks=true,
  urlcolor=blue,
  linkcolor=blue,
  citecolor=darkgreen,
  }

\title{Informe de proyecto: \\ Implementación del algoritmo de rectificación de Loop \& Zhang}
\author{Antonio Álvarez Caballero \\ Alejandro García Montoro \\
    \href{mailto:analca3@correo.ugr.es}{analca3@correo.ugr.es} \\
    \href{mailto:agarciamontoro@correo.ugr.es}{agarciamontoro@correo.ugr.es}}
\date{}
%%%%%%%%%%%%%%%%% FIN PREAMBULO %%%%%%%%%%%%%%%%%%%%%%%%%%%

\begin{document}

  \maketitle

  \section{Descripción del problema}

    La rectificación de imágenes es el proceso de aplicar dos homografías a un par
    de imágenes cuya geometría epipolar conocemos, para que las líneas epipolares
    queden horizontales y paralelas entre sí. El método que proponen \emph{Loop \& Zhang}
    es totalmente geométrico, no precisa de conocimiento de las cámaras.
    Se consigue descomponiendo la homografía en una proyección, una transformación
    euclídea y una cizalla, con especial cuidado en minimizar la distorsión.

    La motivación de la rectificación es simple: es más fácil buscar correspondencias
    cuando las imágenes están rectificadas, ya que sólo hay que buscar en la línea
    epipolar, siendo la distancia que separa ambas correspondencias la \emph{disparidad}.
    Esta distancia está relacionada con la profundidad, por lo que se puede reconstruir
    la profundidad sin noción alguna de cámaras.

    \section{Enfoque de la implementación y detalles de eficiencia}

      La implementación ha tenido un enfoque totalmente funcional, nada de orientación
      a objetos. Se han calculado las transformaciones por separado.

      \subsection{Transformación proyectiva}
      Proyectiva

      \subsection{Euclídea}
      Euclídea

      \subsection{Cizalla}

    \section{Experimentos realizados}

      Lola es muy guapa y adorable. V impone.

    \section{Valoración de resultados}

      Todo se ve de lujo

    \section{Conclusiones}

      Este algoritmo es pro. Mejor que el de OpenCV.

\end{document}
